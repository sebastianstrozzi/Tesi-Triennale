\chapter{Sull'Incompletezza di PA}
		[bla bla bla]
		
		Nel primo capitolo è stata data una dimostrazione della consistanza dell'Aritmetica di Peano, tuttavia i teoremi di Godel escludono la possibilità che la consistenza della teoria elementare dei numeri possa essere dimostrata soltanto con tecniche della stessa.
		Ogni parte della dimostrazione appena data, fatta eccezione per il teorema di accessibilità, è enunciata e dimostrata all'interno di PA. La logica deduzione è che quindi quel qualcosa di indimostrabile debba trovarsi proprio nel teorema. Questo sfrutta sostanzialmente il lemma sugli ordinali limite e il principio di induzione, che però è certamente lecito all'interno della teoria che lo formalizza.
		Gli altri due lemmi non costituiscono un problema in quanto il primo, quello sugli spazi topologici, è soltanto un artificio per velocizzare le dimostrazioni successive che, altrimenti, richiederebbero qualche considerazione in più. Il secondo invece, a conti fatti, resta nell'ambito del finito e perciò interamente dimostrabile in PA.
		Il lemma sugli ordinali limite invece, sfrutta prima una considerazione sulle successioni infinite, e cioè l'esistenza necessaria di un qualche $\gamma = \alpha_{n}$ tra $\beta$ e $\alpha$, poi il principio di \emph{induzione transfinita}. Sfruttando una distinzione per casi simile a quella presente nel teorema di accessibilità, si sarebbe potuto esibire esplicitamente un $\gamma_{\beta}$, dipendente da $\beta$, tale da essere $\beta < \gamma_{\beta} < \alpha$ e far parte di una successione il cui limite fosse proprio $\alpha$.
		Ciò che resta è l'induzione transfinita fino ad $\epsilon_{0}$ che quindi si rivela indimostrabile 
		
		(è il principio di induzione transfinita ad essere indimostrabile? o cosa? perchè gentzen parla di induzione transfinita fino a epsilon0... non è che esiste(?) qualche induzione transfinita intermedia che è effettivamente dimostrabile? Controlla cosa intende per "induzione transf fino a epsilon0"....)
		
		
		The impossibility of proving transfinite induction up to the ordinal 
		number E~ with elementary number-theoretical techniques may be inferred 
		indirectly from the following two facts: 
		1. Godel’s theorem:
		2. The consistency of elementary number theory has been proved by 
		applying transfinite induction up to e0 , together with exclusively elementary 
		number-theoretical techniquesg3.
		
	%\section{Un Primo Esempio di Incompletezza}
		%indecidibilità di Teorema di Accessibilità in PA.
		
	\section{Goodstein e Ramsey}
		motivazioni, dimostrazioni e sospetti di indecidibilità.
		
	\section{Indecidibilità}
		dimostrazioni di indecidibilità.
		
%	\section{Osservazioni Conclusive?}