
\selectlanguage{italian}
\addcontentsline{toc}{chapter}{Bibliografia}
	\begin{thebibliography}{9}
    	
    	% Capitolo 1
    	\bibitem{Gentzen4} \textsc{Gentzen}, G. (1936). \emph{Die Widerspruchsfreiheit der reinen Zahlentheorie -- Mathematische Annalen, 112}. tradotto in: \textsc{Sazabo}, M. E. (1969, ed.). \emph{The Collected Paper of Gerhard Gentzen}. Amsterdam: North-Holland.
    	
    	\bibitem{Gentzen8} \textsc{Gentzen}, G. (1938). \emph{Neue Fassung des Widerspruchsfreiheitsbeweises f\"ur die reine \mbox{Zahlentheorie}}. tradotto in: \textsc{Sazabo}, M. E. (1969, ed.). \emph{The Collected Paper of Gerhard Gentzen}. Amsterdam: North-Holland.
    	
    	\bibitem{Takeuti} \textsc{Takeuti}, G. (1987). \emph{Proof Theory -- Studies in logic and the foundation of mathematics}, V.81. Amsterdam: North-Holland.
    	
    	\bibitem{GodelPoint} \textsc{Horská}, A. (2014). \emph{Where is the \Godel-Point hiding -- Gentzen’s Consistency Proof of 1936 and his Representation of Constructive Ordinals}. Springer.
    	
    	\bibitem{Feferman1} \textsc{Feferman}, S. (2006). \emph{The impact of the Incompleteness Theorems on Mathematics}. Notices American Mathematical Society.
    	
    	\bibitem{LogicaZG} \textsc{Berto}, F. (2007). \emph{Logica da Zero a \Godel}. Editori Laterza.
    	% Capitolo 2
    	%\bibitem{ref} \textsc{Feferman}, N. e \textsc{Cognome2}, N2.~N2. (Anno). \emph{The impact of the incompleteness theorems on mathematics}. Casa editrice.
    	
    	% Altre
    	%\bibitem{ref2} \textsc{Cognome}, N. e \textsc{Cognome2}, N2.~N2. (Anno). \emph{Titolo -- Sottotitolo}. Casa editrice.
    	    	
	\end{thebibliography}
