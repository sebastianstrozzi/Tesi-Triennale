% Pacchetti
	\usepackage[T1]{fontenc}
	\usepackage[utf8]{inputenc}
	\usepackage{setspace}					% Richiesto per frontespizio
	\usepackage[italian, english]{babel}
	\usepackage{amsmath}
	\usepackage{amssymb}
	\newcommand\hmmax{0} 					% default 3
	\newcommand\bmmax{0} 					% default 4
	\usepackage{bm}
	\usepackage{mathrsfs}
	\usepackage{amsthm}
	\usepackage{bussproofs}					% figure di inferenza
	\usepackage{bussproofs-extra}
	\usepackage{bm}							% Simboli in bold
	\usepackage{comment} 		            % Usa \begin{comment} \end{comment}
	\usepackage{braket} 					% Parentesi
	\usepackage{MnSymbol}
	\usepackage{hyperref} 					% Riferimenti
	\usepackage[capitalise]{cleveref} 		% Riferimenti
	\usepackage{emptypage} 					% Migliora pagine vuote
	\usepackage{enumitem} 					% Liste
	\usepackage{fancyhdr} 					% Modifica stile di pagina
	\usepackage{float}						% ????
	\usepackage
	[a4paper,
	width=160mm,
	top=25mm,
	bottom=25mm,
	bindingoffset=6mm
	]{geometry} 							%Margini
	\usepackage{graphicx}					% Immagini
	\usepackage{caption}
	\usepackage{subfig}						% Immagini
	\usepackage{mathtools}					
	\usepackage{nicefrac}					% Frazioni carine
	\usepackage{stmaryrd}					% ????
	\usepackage{tikz-cd} 					% ????
	\usepackage{titlesec}       			% Modifica il formato dei titoli
	\usepackage{verbatim}					
	\usepackage[swapnames]{frontespizio}	% Non credo di starlo usando
	\usepackage{lipsum} 					% ????
	\usepackage{afterpage}
	
	\newcommand\blankpage{%
		\null
		\thispagestyle{empty}%
		\addtocounter{page}{-1}%
		\newpage}


% Environment Abstract che non c'è nella classe Book
	\newenvironment{abstract}{
		\cleardoublepage \null \vfill 
		\begin{center}
			\bfseries\abstractname
		\end{center}
	}
	{\vfill\null} 						


% Usa fonts più piccoli per le didascalie
	\captionsetup{font=small, labelfont={sf, bf}} 


% Formattazione titoli
	\titleformat{\chapter}
	{\Huge \normalfont \bfseries}{\thechapter}{1em}{}


% Stile di Pagina
	\pagestyle{fancy}
	\fancyhf{}
	\fancyhead[LE]{\bfseries \leftmark}
	\fancyhead[RO]{\bfseries \rightmark}
	\fancyhead[RE, LO]{\bfseries \thepage}

	\setlength{\headheight}{14.5pt}

	\renewcommand\chaptermark[1]{\markboth{\thechapter.\space#1}{}} 
	\renewcommand\sectionmark[1]{\markright{\thesection.\ #1}} 


% Usa stile di pagina vuota invece di plain, rimuove la numerazione dalle pagine dei capitoli
	\makeatletter
	\let\ps@plain\ps@empty
	\makeatother


% Impostazioni dei colori per i riferimenti
	\hypersetup{ 
		colorlinks=false, 
		citecolor=Violet,
		linkcolor=Red,
		urlcolor=Blue
	}


% Comandi Generici
	\def\Egrave{\MakeUppercase{è}\ }									% E accentata
	\def\apertevirg{‘‘}
	\def\chiusevirg{''}
	\newcommand{\virg}[1]{\apertevirg #1\chiusevirg}
	\def\Godel{G\"odel}
	\def\N{\ensuremath{\mathbb{N}}}										% simbolo Naturali
	\def\defeq{\ensuremath{\stackrel{\text{def}}{=}}}					% uguale per def
	\def\nn{\ensuremath{\neg}}											% not
	\def\ee{\ensuremath{\land}}											% and
	\def\oo{\ensuremath{\lor}}											% or
%	\def\allora{\ensuremath{\supset}}									% implicazione 1
	\def\allora{\ensuremath{\Rightarrow}}								% implicazione 2
	\def\perogni{\ensuremath{\forall}}									% per ogni
	\def\esiste{\ensuremath{\exists}}									% esiste
	
	% Lettere gotiche
	\def\gothD{\ensuremath{\mathfrak{D}}}
	\def\gothU{\ensuremath{\mathfrak{U}}}
	\def\gothB{\ensuremath{\mathfrak{B}}}
	\def\gothF{\ensuremath{\mathfrak{F}}}
	


% Comandi per il cap.1
	\def\seq{\ensuremath{\bm{\rightarrow}}}								% simbolo sequente
	\def\emptyseq{\ensuremath{\  \seq \; }}								% sequente vuoto
	\newcommand{\CJ}[1]{\ensuremath{\bm{CJ}-#1}}						% CJ-"qualcosa"	
	\newcommand{\Sordinale}[1]{\ensuremath{\mathfrak{S}_{#1}}}			% insiemi ordinali
	\newcommand{\tetrazione}[2]{\ensuremath{#1\uparrow\uparrow#2}}		% tetrazione
	\newcommand{\ordsum}[2]{\ensuremath{#1\,\texttt{\#}\,#2}}			% somma ordinali
	\newcommand{\ordof}[1]{\ensuremath{\sigma(\mathcal{#1})}}			% o(dimostrazione)
	\def\proofsym{\ensuremath{\mathcal{D}}}								% simbolo dim
	\def\redproof{\ensuremath{\mathcal{D\,'}}}							% dim ridotta
	\newcommand{\redordof}[1]{\ensuremath{\sigma(\mathcal{#1\,'})}}		% o(D')
	\def\cutC{\ensuremath{\mathcal{C}}}									% lettera Cut
	\def\teoriaS{\ensuremath{\mathscr{S}}}
	
	% puntini inferenza
	\def\goon{\ensuremath{\ddots\,\vdots\,\udots}}
	\def\inferencedots{
		\noLine
		\UnaryInfC{\goon}
		\def\extraVskip{0pt}
		\noLine
		\UnaryInfC{\vdots}
		\def\extraVskip{2pt}
		}
	\newcommand{\longInfdots}[1]{
		\noLine
		\UnaryInfC{\vdots}
		\RightLabel{\quad \; #1}
		\def\extraVskip{0pt}
		\noLine
		\UnaryInfC{\vdots}
		}


% Stile dei Teoremi
	\theoremstyle{plain}
	\newtheorem{teo}{Teorema}[section]									% prima [chapter]
	\newtheorem{prop}[teo]{Proposizione}
	\newtheorem{lemma}[teo]{Lemma}
	\newtheorem{corollary}{Corollario}[teo]
	\theoremstyle{definition}
	\newtheorem{defin}[teo]{Definizione}
	\newtheorem{eg}[teo]{Esempio}